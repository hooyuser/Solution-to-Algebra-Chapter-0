% Compile with PdfLaTeX

\documentclass[12pt,letterpaper,boxed]{hmcpset}

% set 1-inch margins in the document
\usepackage[margin=1in]{geometry}

% include this if you want to import graphics files with /includegraphics

\usepackage{graphicx}
\usepackage{amssymb}
\usepackage[all]{xy}

\usepackage{setspace}

\renewcommand{\baselinestretch}{1.05} 

% info for header block in upper right hand corner
\name{Huyi Chen}
\updatedate{2018/03/31}

\begin{document}
		
\problemlist{\textbf{Algebra, Chapter 0}\\By Paolo Aluffi}


\begin{problem}[2.1]
	One can associate an $n\times n$ matrix $M_\sigma$ with a permutation $\sigma \in S_n$, by
	letting the entry at $(i, \sigma(i))$ be 1, and letting all other entries be 0. For example,
	the matrix corresponding to the permutation
	\[
	\sigma=\left(
	\begin{matrix}	
	1 & 2 & 3\\	
    3 & 1 & 2	
	\end{matrix}
	\right)\in S_3
	\]
	would be 
	\[
	M_\sigma=\left(
	\begin{matrix}	
	0 & 0 & 1\\	
	1 & 0 & 0\\	
	0 & 1 & 0
	\end{matrix}
	\right)
	\]
	Prove that, with this notation,
	\[
	M_{\sigma\tau}=M_{\sigma}M_{\tau}
	\]
	for all $\sigma,\tau\in S_n$, where the product on the right is the ordinary product of matrices.
\end{problem}
\begin{solution}
	With Kronecker delta function	
    \[
    \delta _{{i,j}}=
    	\begin{cases}
    	0{\text{\quad if }}i\neq j,\\
    	1{\text{\quad if }}i=j,
    	\end{cases}
    \]
	the entry at $(i,j)$ of the matrix $M_{\sigma\tau}$ can be written as
	\[
	(M_{\sigma\tau})_{i,j}=\delta_{\tau(\sigma(i)),j}
	\]
	and the entry at $(i,j)$ of the matrix $M_{\sigma}M_{\tau}$ can be written as
	\[
	(M_{\sigma}M_{\tau})_{i,j}=\sum_{k=1}^{n}(M_{\sigma})_{i,k}(M_{\tau})_{k,j}=\sum_{k=1}^{n}\delta_{\sigma(i),k}\cdot\delta_{\tau(k),j}=\sum_{k=1}^{n}\delta_{\sigma(i),k}\cdot\delta_{k,\tau^{-1}(j)}=\delta_{\sigma(i),\tau^{-1}(j)}.
	\]
	Note that
	\[
	\tau(\sigma(i))=j\iff \sigma(i)=\tau^{-1}(j),
	\]
    we see $M_{\sigma\tau}=M_{\sigma}M_{\tau}$
    for all $\sigma,\tau\in S_n$.
    
   
\end{solution}
~\\

\begin{problem}[2.2]
	Prove that if $d \le n$, then $S_n$ contains elements of order $d$.
\end{problem}
\begin{solution}
	The cyclic permutation 
	\[
	\sigma=(1\;2\;3\cdots d)
	\]
    is an element of order $d$ in $S_n$.
\end{solution}

~\\

\begin{problem}[2.3]
	For every positive integer $n$ find an element of order $n$ in $S_\mathbb{N}$.
\end{problem}
\begin{solution}
The cyclic permutation 
\[
\sigma=(1\;2\;3\cdots n)
\]
is an element of order $d$ in $S_n$.
\end{solution}

~\\

\begin{problem}[2.4]	
	Define a homomorphism $D_8 \rightarrow S_4$ by labeling vertices of a square, as we did
	for a triangle in \textsection2.2. List the 8 permutations in the image of this homomorphism.
\end{problem}
\begin{solution}
	The image of $n$ rotations under the homomorphism are
	\[
	\sigma_1=e_{D_8},\;\sigma_2=(1\;2\;3\;4),\;\sigma_3=(1\;3)(2\;4),\;\sigma_4=(1\;4\;3\;2).
	\]
	The image of $n$ reflections under the homomorphism are
	\[
	\sigma_5=(1\;3),\;\sigma_6=(2\;4),\;\sigma_7=(1\;2)(3\;4),\;\sigma_8=(1\;4)(3\;2).
	\]
\end{solution}

~\\
\begin{problem}[3.1]	
	Let $\varphi : G\rightarrow H$ be a morphism in a category $\mathsf{C}$ with products. Explain why
	there is a unique morphism
	\[
	(\varphi\times\varphi) : G \times G \longrightarrow H \times H .
	\]
	(This morphism is defined explicitly for $\mathsf{C} = \mathsf{Set}$ in \textsection 3.1.)
\end{problem}
\begin{solution}
	By the universal property of product in $\mathsf{C}$, there exist a unique morphism $(\varphi\times\varphi) : G \times G \longrightarrow H \times H$ such that the following diagram commutes.
    \[\xymatrix{
    	G\ar[r]^{\varphi} & H \\
    	G \times G\ar[u]^{\pi_G}\ar[d]_{\pi_G}\ar[r]^{\varphi\times\varphi} &  H\times H\ar[u]_{\pi_H}\ar[d]^{\pi_H} \\
    	G\ar[r]^{\varphi} & H 
    }\]
\end{solution}

~\\

\begin{problem}[3.2]	
	Let $\varphi : G\rightarrow H, \psi : H \rightarrow K$ be morphisms in a category with products, and
	consider morphisms between the products $G\times G, H\times H, K\times K$ as in Exercise 3.1.
	Prove that
	\[
	(\psi\varphi) \times(\psi\varphi)=(\psi \times \psi)(\varphi\times \varphi) .
	\]
	(This is part of the commutativity of the diagram displayed in \textsection 3.2.)
\end{problem}
\begin{solution}
	By the universal property of product in $\mathsf{C}$, there exist a unique morphism 
	\[
	(\psi\varphi) \times(\psi\varphi):G\times G\rightarrow K\times K
	\] 
	such that the following diagram commutes.
	\[\xymatrix{
		G\ar[rr]^{\psi\varphi} && H \\
		G \times G\ar[u]^{\pi_G}\ar[d]_{\pi_G}\ar[rr]^{	(\psi\varphi) \times(\psi\varphi)} &&  H\times H\ar[u]_{\pi_H}\ar[d]^{\pi_H} \\
		G\ar[rr]^{\psi\varphi} && H 
	}\]
    As the following commuting diagram tells us the composition 
    \[
    (\psi \times \psi)(\varphi\times \varphi):G\times G\rightarrow K\times K
    \]
    can make the above diagram commute,
	\[\xymatrix{
		G\ar[r]^{\varphi}\ar@/^1.6pc/[rr]^{\psi\varphi} & H\ar[r]^{\psi} & K \\
		G \times G\ar[u]^{\pi_G}\ar[d]_{\pi_G}\ar[r]^{\varphi\times\varphi} &  H\times H\ar[u]^{\pi_H}\ar[d]_{\pi_H}\ar[r]^{\psi\times\psi} &  K\times K \ar[u]^{\pi_K}\ar[d]_{\pi_K}\\
		G\ar[r]^{\varphi}\ar@/_1.6pc/[rr]_{\psi\varphi} & H \ar[r]^{\psi} & K
	}\]
    there must be $(\psi\varphi) \times(\psi\varphi)=(\psi \times \psi)(\varphi\times \varphi)$.
    
\end{solution}

~\\

\begin{problem}[3.3]	
	Show that if $G, H$ are abelian groups, then $G \times H$ satisfies the universal property for coproducts in $\mathsf{Ab}$.
\end{problem}
\begin{solution}
	Define two monomorphisms:
	\[
	i_G:G\longrightarrow G\times H,\;a\longmapsto (a,0_H)
	\]
	\[
	i_H:H\longrightarrow G\times H,\;b\longmapsto (0_G,b)
	\]
	We are proving that for any two homomorphisms $g:G\rightarrow M$ and $h:H\rightarrow M$ in $\mathsf{Ab}$, the map
	\[
	\begin{aligned}
	\varphi:\quad & G\times H\longrightarrow M,\\
	         & (a,b)\longmapsto g(a)+h(b)
	\end{aligned}
    \]
    is a homomorphism and makes the following diagram commute. 
	\[\xymatrix{
		G\ar[rd]^{g}\ar[d]_{i_G}  \\
		G \times H\ar[r]^{\varphi} &  M\\
		H\ar[ru]_{h}\ar[u]^{i_H}  
	}\]
	Exploiting the fact that $g,h$ are homomorphisms and $M$ is an abelian group, it is easy to check that $\varphi$ preserves the addition operation
	\[
	\begin{aligned}
	\varphi((a_1,b_1)+(a_2,b_2))&=\varphi((a_1+a_2,b_1+b_2))\\
	&=g(a_1+a_2)+h(b_1+b_2)\\
	&=(g(a_1)+g(a_2))+(h(b_1)+h(b_2))\\
	&=(g(a_1)+h(b_1))+(g(a_2)+h(b_2))\\
	&=g(a_1+b_1)+h(a_2+b_2)\\
	&=\varphi((a_1,b_1))+\varphi((a_2,b_2))
	\end{aligned}
	\]
	and the diagram commutes
	\[
	\varphi\circ i_G(a)=\varphi((a,0_H))=g(a)+h(0_H)=g(a)+0_M=g(a),
	\]
	\[
	\varphi\circ i_H(b)=\varphi((0_G,b))=g(0_G)+h(b)=0_M+h(b)=h(b).
	\]
	To show the uniqueness of the homomorphism $\varphi$ we have constructed, suppose a homomorphism $\varphi'$ can make the diagram commute. Then we have
	\[
	\varphi'((a,b))=\varphi'((a,0_H)+(0_G,b))=\varphi'(i_G(a))+\varphi'(i_H(b))=g(a)+h(b)=\varphi((a,b)),
	\]
	that is $\varphi'=\varphi$. Hence we show that there exist a unique homomorphism $\varphi$ such that the diagram commutes, which amounts to the universal property for coproducts in $\mathsf{Ab}$.
	
\end{solution}

~\\

\begin{problem}[3.3]	
	Prove that $\mathbb{Q}$ is not the direct product of two nontrivial groups.
\end{problem}
\begin{solution}
	\[\xymatrix{
		G\ar[rd]^{g}\ar[d]_{i_G}  \\
		G \times H\ar[r]^{\varphi} &  \mathbb{Q}\\
		H\ar[ru]_{h}\ar[u]^{i_H}  
	}\]
    Consider the additive group of rationals $(\mathbb{Q},+)$. Assume the product $G\times H=\{(a,b)|a\in G,b\in H\}$ is isomorphic to $(\mathbb{Q},+)$. Note that $\{e_G\}\times H$ and $G\times \{e_H\}$ are subgroups in $G\times H$ and there intersection is trivial group $\{e_G\}\times \{e_H\}$. The commutative diagram implies 
    \[
    \varphi(\{e_G\}\times H)=\varphi(i_H(H))=h(H),
    \]
    \[
    \varphi(G\times \{e_H\})=\varphi(i_G(G))=g(G).
    \]
    It is easy to check bijection $\varphi$ satisfies $\varphi(A\cap B)=\varphi(A)\cap\varphi (B)$. Hence we have
    \[
    \varphi(\{(e_G,e_H)\})=\varphi(\{e_G\}\times H\cap G\times \{e_H\})=\varphi(\{e_G\}\times H)\cap \varphi(G\times \{e_H\})=h(H)\cap g(G)=\{0\}.
    \] 
    Suppose both $g(G)$ and $h(H)$ are nontrivial groups. If $\dfrac{p}{q}\in h(H)-\{0\}$ and $\dfrac{r}{s}\in g(G)-\{0\}$, there must be 
    \[
    rp=rq\cdot\dfrac{p}{q}=ps\cdot\dfrac{r}{s}\in h(H)\cap g(G). 
    \]
    Since $rp\ne0$, it leads to a contradiction. Thus we can assume $g(G)$ is a trivial group. According to the dual commutative diagram,
    \[\xymatrix{
    	G\ar[rrd]^{g}\ar[d]_{i_G}  \\
    	G \times H &  &\mathbb{Q}\ar[ll]_{\varphi^{-1}}\\
    	H\ar[rru]_{h}\ar[u]^{i_H}  
    }\]
    we see that for all $a\in G$,
    \[
    (a,e_H)=i_(a)=\varphi^{-1}(g(a))=\varphi(0)=(e_G,e_H)\implies a=e_G.
    \]
    that is, $G$ is a trivial group. Therefore, we have shown $(\mathbb{Q},+)$ will never be isomorphic to the direct product of two nontrivial groups.
\end{solution}

~\\


\newpage
\newpage
\newpage
 Assume
\[
\varphi(a_1,b_1)=g(a_1)+h(b_1)=1,
\]
By induction we can show for all $p\in\mathbb{N}$.
\[
\varphi(a_1^p,b_1^p)=pg(a_1)+ph(b_1)=p,
\]
For all $q\in\mathbb{N}-\{0\}$, there exist unique $(c_q,d_q)\in G\times H$ such that
\[
\varphi(c_q,d_q)=g(c_q)+h(d_q)=\frac{1}{q},
\]
namely
\[
\varphi(c_q^q,d_q^q)=q\varphi(c_q,d_q)=1=\varphi(a_1,b_1)\implies (c_q^q,d_q^q)=(a_1,b_1).
\]
Denote $c_q=a_1^{\frac{1}{q}},\;d_q=b_1^{\frac{1}{q}}$. 
\[
\varphi([(a_1^{\frac{1}{q}})^p]^q,[(b_1^{\frac{1}{q}})^p]^q)=\varphi((a_1^{\frac{1}{q}})^{pq},(b_1^{\frac{1}{q}})^{pq})=pq\varphi((a_1^{\frac{1}{q}},b_1^{\frac{1}{q}}))=pq\frac{1}{q}=\varphi(a_1^p,b_1^p)
\]
implies
\[
[(a_1^{\frac{1}{q}})^p]^q=a_1^p,\;[(b_1^{\frac{1}{q}})^p]^q=b_1^p
\]
Denote $(a_1^{\frac{1}{q}})^p=(a_1^p)^{\frac{1}{q}}=a_1^{\frac{p}{q}},\;(b_1^{\frac{1}{q}})^p=(b_1^p)^{\frac{1}{q}}=b_1^{\frac{p}{q}}$. 
Then 
\[
g(a_1^{\frac{p}{q}})=pg(a_1^{\frac{1}{q}})=\frac{p}{q}g(a_1)
\]
\[
h(b_1^{\frac{p}{q}})=ph(b_1^{\frac{1}{q}})=\frac{p}{q}h(b_1).
\] 
For all $p\in \mathbb{N}$, if $h(b_1)\ne 0$,   
\[
\begin{aligned}
p&=\varphi\left(a_1^p,b_1^p\right)\\
&=(p+1)g(a_1)+\left(p-\frac{g(a_1)}{h(b_1)}\right)h(b_1)\\
&=g(a_1^{p+1})+h\left(b_1^{p-\frac{g(a_1)}{h(b_1)}}\right)\\
&=\varphi\left(a_1^{p+1},b_1^{p-\frac{g(a_1)}{h(b_1)}}\right)
\end{aligned}
\]
Therefore, $a_1^p=a_1^{p+1}\implies a_1=e_G$





Hence for all $\dfrac{p}{q}\in\mathbb{Q}$, it holds that
\[
\frac{p}{q}=p\varphi(a_1^{\frac{1}{q}},b_1^{\frac{1}{q}})=\varphi(a_1^{\frac{p}{q}},b_1^{\frac{p}{q}})=\frac{p}{q}g(a_1)+\frac{p}{q}h(b_1).
\]
Suppose
\[
\varphi(a_1,e_H)=\dfrac{r}{s}=\varphi(c_{s}^{r},d_{s}^{r}),
\]
which indicates
\[
(a_1,e_H)=(c_{s}^{r},d_{s}^{r})\implies d_{s}^{r}=e_H.
\]
Likewise, we suppose 
\[
\varphi(e_G,d_q^p)=\dfrac{m}{n}=\varphi(c_{n}^{m},d_{n}^{m}),
\]
and get $c_{n}^{m}=e_G$.
\[
\varphi(c_q^p,d_q^p)=\varphi(c_q^p,e_H)+\varphi(e_G,d_q^p)=rg(c_s)+mh(d_n)
\]

\end{document}
